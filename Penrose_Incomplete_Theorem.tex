\documentclass[12pt, a4paper]{report}
    \usepackage[utf8]{inputenc}
    \usepackage[T1]{fontenc}
    \usepackage[urw-garamond]{mathdesign} %for using math with garamond font
    \usepackage{mathtools}
    \usepackage{titlesec}
    \usepackage{amsmath, xparse} %for cross-referencing math equations
    \usepackage[mathscr]{eucal}
    %\usepackage{amsfonts}
    \usepackage{amsthm}
    %\usepackage{amssymb}
    %\let\oldemptyset\emptyset
    %\let\emptyset\varnothing
    \usepackage{bm} %for using bold font letters in inline mathematical expressions
    \usepackage{xcolor}
    \usepackage{graphicx}
    \usepackage{graphics}
    \usepackage{subcaption}
    \usepackage{array}
    \usepackage{epstopdf}
    \usepackage{titlesec} %for numbering of subsubsection
    \usepackage{parskip}
    \usepackage{enumitem} %for enumerations of lists with numbers or alphabets and so on.......
    \usepackage[margin=2cm, includefoot]{geometry}
    \usepackage[labelfont={small, bf}, textfont={small, it}]{caption} %used for making the captions in images small and italicized
    \usepackage{cancel}
    
    \usepackage{cleveref}
    \crefname{equation}{Equation}{Equations}
    
    \usepackage{tikz}
    \usetikzlibrary{arrows}

    \makeatletter
    \newcommand\@erelb@r[1]{%
    \mathrel{\tikz[baseline=-.5ex]\draw[#1] (0,0)--(0.3,0);}
    }
    % 0 is for nothing
    % 1 is for arrowhead
    % 2 is for bar
    % 3 is for both
    \newcommand{\erelbar}[1]{\@erelbar#1}
    \def\@erelbar#1#2{%
    \ifcase\numexpr#1*4+#2\relax
        \@erelb@r{-}\or     % 00
        \@erelb@r{->}\or    % 01
        \@erelb@r{-|}\or    % 02
        \@erelb@r{->|}\or   % 03
        \@erelb@r{<-}\or    % 10
        \@erelb@r{<->}\or   % 11
        \@erelb@r{<-|}\or   % 12
        \@erelb@r{<->}\or   % 13
        \@erelb@r{|-}\or    % 20
        \@erelb@r{|->}\or   % 21
        \@erelb@r{|-|}\or   % 22
        \@erelb@r{|<->|}\or % 23
        \@erelb@r{|<-}\or   % 30
        \@erelb@r{|<->}\or  % 31
        \@erelb@r{|<-|}\or  % 32
        \@erelb@r{|<->|}    % 33
    \else
        \@wrong
    \fi
    }
    \makeatother
    
    % Custom definition for footnote symbols
    \makeatletter
    \def\@fnsymbol#1{\ensuremath{\ifcase#1\or *\or \dagger\or \ddagger\or
    \mathsection\or \mathparagraph\or \|\or **\or \dagger\dagger
    \or \ddagger\ddagger \else \@ctrerr\fi}}
    \makeatother

    % Packages for title page and section formatting
    \usepackage{titling}
    \usepackage{sectsty}
    \usepackage{titlesec}

    % Page Layout
    \textwidth 165mm    % also can be set to 155mm///// 158mm
    \textheight 238mm   % standard
    \topmargin -10mm

    \oddsidemargin 0.1cm
    \evensidemargin -0.07cm
    \graphicspath{ {/images/} }

    % Stopping Figures from Using a Whole Page
    \renewcommand{\topfraction}{0.95}
    \renewcommand{\textfraction}{0.05}
    \renewcommand{\floatpagefraction}{0.85}
    \renewcommand{\baselinestretch}{1.5}
    \renewcommand*\descriptionlabel[1]{\hspace\leftmargin$#1$}
    \renewcommand{\thefootnote}{\fnsymbol{footnote}} %for using footnotes with symbols rather than numbers
    \renewcommand{\bibname}{References} %to rename bibliography section to "References"
    
    % Set section font size and type, and adjust spacing
    \titleformat{\section}
        {\Large\bfseries\sffamily} % Customize font size and type
        {\thesection}{1em}{}
    \titlespacing*{\section}
        {0pt}{3.0ex plus 1ex minus .2ex}{2.0ex plus .2ex}

    \titleformat{name=\section, numberless}
        {\Large\bfseries\sffamily} % Customize font size and type
        {}{0pt}{}
        
    % Optionally, customize subsection and subsubsection as well
    \titleformat{\subsection}
        {\large\bfseries\itshape}
        {\thesubsection}{1em}{}
    \titlespacing*{\subsection}
        {0pt}{2.0ex plus 1ex minus .2ex}{1.0ex plus .2ex}

    \titleformat{\subsubsection}
        {\normalsize\bfseries\scshape}
        {\thesubsubsection}{1em}{}
    \titlespacing*{\subsubsection}
        {0pt}{1.5ex plus 1ex minus .2ex}{1.0ex plus .2ex}
       
    %defines a new format for the theorems (such as top space, bottom space, text title type, etc.)
    \newtheoremstyle{bfnote}%
        {16.0pt plus 2.0pt minus 4.0pt}{\topsep} %\topsep is defined as {8.0pt plus 2.0pt minus 4.0pt}
        {\itshape}{}
        {\bfseries}{.}
        { }{\thmname{#1}\thmnumber{ #2}\thmnote{ (#3)}}

    \theoremstyle{bfnote}
    
    \newtheorem{example}{Example}[section] %for enumerating examples within a chapter
    \newtheorem{theorem}{Theorem}[section]
    \newtheorem{corollary}{Corollary}[theorem]
    \newtheorem{proposition}{Proposition}[section]
    \newtheorem{definition}{Definition}[section]

    \newcommand{\underit}[1]{\textit{\underbar{#1}}}
    \newcommand{\chits}{\raisebox{0.5ex}{$\chi$}} %typeset chi

    %\numberwithin{equation}{section}
    
    \renewcommand{\thesection}{\arabic{section}} %to remove 0.x in numbered sections
    %\renewcommand{\theorem}{\arabic{section}.\arabic{theorem}}


\begin{document}
%============================================BEGIN TITLE PAGE===============================================
\begin{titlepage}
    \begin{center}
    %syntax for entering line \line(slope){length} double backslash for change of line. For changing
    %line one can even leave a space. 
    \line(1,0){300}\\
    %for increasing the space between the line and the title [spacing]	
    [0.2in]
    \normalsize{\emph {Seminar Report}}\\	
    \Huge{\bfseries Penrose Incompleteness Theorem}\\
    [-0.12in]
    \line(1,0){300}\\
    
    \vspace{3in}
    
    \LARGE{\bfseries {Prabha Shankar}}\\
    \small{Masters Student, Mathematical Physics}\\
    \small{(Department of Physics, Leipzig University)}\\
    
    \vspace{3cm}
    \normalsize{\emph {Under the supervision of}}\\
    \large{\bfseries {Prof. Dr. Stephan Czimek}}\\
    \small{(Department of Mathematics, Leipzig University)}\\
    
    \end{center}
\end{titlepage}

%=================TABLE OF CONTENTS AND LIST OF FIGURES=======================
\section*{Introduction}
This is the placeholder for introduction.

\section{Double Null Hypersurfaces and Foliations}
This is the first chapter

\section{Null Structure Equations}
This is the second chapter 

\section{Condition for Regularity of Null Hypersurfaces}
This is the third chapter

\section{Causality for Spacetimes with Trapped Surfaces}
Recall the geometric construction in the previous sections. If $S$ is a closed
2-dimensional surface in a globally hyperbolic time-orientable spacetime
$\left(\mathcal{M}, g\right)$ and $C$ and \textit{\underbar{C}} be future
outgoing and incoming null geodesic congruence normal to $S$ respectively. Then:
\begin{equation}
    \partial\hspace{-0.3em}\mathcal{J}^+\left(S\right) \subset C \cup \textit{\underbar{C}}.
\end{equation}
We always have:
\begin{equation}
    C \cup \textit{\underbar{C}} \subset \mathcal{J}^+\left(S\right).
\end{equation}
However, it is not always the case that:
\begin{equation}\label{eq:BCnotalwaystrue}
    C \cup \textit{\underbar{C}} \subset \partial\hspace{-0.3em}\mathcal{J}^+\left(S\right).
\end{equation}
The reason for this is in case $C \cup \textit{\underbar{C}}$ lies in the
interior of the future of $S$, i.e., $\mathcal{J}^+\left(S\right)$ implies
$\exists$ timelike curves connecting points on $C \cup \textit{\underbar{C}}$ to
$S$ and also to a neighborhood of $S$ which is interior of
$\mathcal{J}^+\left(S\right)$.

\textcolor{red}{Insert connecting text here}

\begin{proposition}
    Let $S$ be a closed 2-dimensional manifold surface in a globally hyperbolic
    time oriented spacetime $\left(\mathcal{M}, g\right)$. Let $C$ and
    \textit{\underbar{C}} denote the (future) outgoing and incoming null
    geodesic congruence normal to $S$. Let $C^*$ and $\textit{\underbar{C}}^*$
    be parts of $C$ and \textit{\underbar{C}} that do not contain any focal
    point. Then, $\partial\hspace{-0.3em}\mathcal{J}^+\left(S\right) \subset C^*
    \cup \textit{\underbar{C}}^*$.
\end{proposition}

\subsection{Trapped Surface}
Assume $\Omega = 1$ on $C \cup \textit{\underbar{C}}$. Consider $L$ and
$\underit{L}$ to be the geodesic vector field of $C$ and $\underit{C}$ with
$\tau$ and \underit{$\tau$} the respective affine parameters such that $S \equiv
\{\tau = 0\} = \{\underit{$\tau$} = 0\} \equiv S_0$.

\subsubsection*{Relationsihp between the area of sections $S_\tau$ and the second fundamental forms $\chits$
and $\underit{\chits}$:} 
The area relates, as we shall see later, the
$\mathrm{det}\cancel{g}$ to $\mathrm{tr}\chits$ and helps deduce the regularity of the
null hypersurfaces. We present the solutions only corresponding to the
hypersurface $C$. The computations are same for \underit{C}.\\
We assume the canonical coordinates $\left(\tau, \theta^1, \theta^2\right)$ on
$C$, where $\left(\theta^1, \theta^2\right) \in \mathcal{U} \subset \mathbb{R}^2$
and $\cancel{g}$ is the induced metric on the sections $S_\tau$ of $C$.

In the following, we use the first variational formula and the Jacobi formula
for the derivative of a matrix determinant:
\begin{equation*}
    \begin{aligned}
        \nabla_L\left(\sqrt{\det{\cancel{g}}}\right) \hspace{0.5em} &= \hspace{0.5em} \frac{1}{2\sqrt{\det\cancel{g}}}\nabla_L\left(\det{\cancel{g}}\right) \hspace{0.5em}= \hspace{0.5em} \frac{1}{2\sqrt{\det\cancel{g}}}\left(\det\cancel{g}\right)\mathrm{tr}\left(\left(\cancel{g}^{-1}\right)^{AB}\nabla_L\cancel{g}_{AB}\right)\\[1em]
                                                                    &= \hspace{0.5em} \frac{\sqrt{\det\cancel{g}}}{2}\mathrm{tr}\left(\left(\cancel{g}^{-1}\right)^{AB}\mathcal{L}_L\cancel{g}_{AB}\right) \hspace{0.5em} = \hspace{0.5em} \frac{\sqrt{\det\cancel{g}}}{2}\mathrm{tr}\left(\left(\cancel{g}^{-1}\right)^{AB}2\chits_{AB}\right)\\[1em]
                                                                    &= \hspace{0.5em} \sqrt{\det\cancel{g}}\hspace{0.4em}\mathrm{tr}\left(\chits\right)
    \end{aligned}
\end{equation*}
We know, 
\begin{equation*}
    \mathrm{Area}\left(S_\tau\right) = \int\limits_{\mathcal{U}} \sqrt{\det\cancel{g}\left(\tau\right)} \: d\theta^1 d\theta^2
\end{equation*}
Thus, 
\begin{equation*}
    \nabla_L\left(\mathrm{Area}\left(S_\tau\right)\right) = \int\limits_{\mathcal{U}}\mathrm{tr}\mathrm{\chits}d\hspace{-0.2em}\mu_{\cancel{g}}.
\end{equation*}
More generally,
\begin{equation}{\label{eq:AreachangealongL}}
    \nabla_{fL}\left(\mathrm{Area}\left(S_\tau\right)\right) = \int\limits_{\mathcal{U}}f\cdot\mathrm{tr}\mathrm{\chits}d\hspace{-0.2em}\mu_{\cancel{g}}\;, \quad \forall f \geq 0 \text{ and } f \in C^\infty\left(S_\tau\right).
\end{equation}
{\bfseries{Interpretation:}} The equations above represent the rate of change of
the second fundamental form $\chits$  and the rate of change of the area of
$S_\tau$ under infinitesimal displacement along the null generators. Therefore,
$\mathrm{tr}\chits$ is also called expansion of $S_\tau$.

\begin{definition}[Trapped Surfaces]
A 2-dimensional surface $S$ in $\left(\mathcal{M}, g\right)$ for which the area
decreases under infinitesimal (arbitrary) displacements along the null
generators of both null geodesics congruences $C \cup \underit{C}$ normal to
$S$.
\end{definition}
If $\left(\mathcal{M}, g\right)$ is globally hyperbolic with a trapped surface,
implies that $C \cup \textit{C}$ bounds the future of S, i.e.,
$\partial\hspace{-0.3em}\mathcal{J}^+\left(S\right) \subset C \cup \underit{C}$.
Hence, it cannot expand in its future. Thus the term \textit{trapped}. It is for
this reason \eqref{eq:BCnotalwaystrue} does not always hold true.\\
In view of this definition and \eqref{eq:AreachangealongL}, for a trapped
surface, the following statements holds:
\begin{equation*}
    \int\limits_{\mathcal{U}}f\cdot\mathrm{tr}\chits\;d\hspace{-0.2em}\mu_{\cancel{g}}\; < \;0\;, \quad \int\limits_{\mathcal{U}}f\cdot\mathrm{tr}\underit{\chits}\;d\hspace{-0.2em}\mu_{\cancel{g}}\; < \;0 \quad \forall f \geq 0 \text{ and } f\in C^\infty\left(S_\tau\right)
\end{equation*}
\textit{Equivalent definition for Trapped Surfaces:} A trapped surface is a
closed 2-dimensional surface $S$ in a Lorentzian manifold $\left(\mathcal{M},
g\right)$ such that:
\begin{equation*}
    \mathrm{tr}\chits\; < \;0\;, \quad \mathrm{tr}\underit{\chits}\; < \;0
\end{equation*}

\subsubsection{Trapped surfaces and focal points:}
A null generators on an incoming null hypersurface $\underit{C}$ contain focal
points. The null expansion of an incoming null hypersurface, in our case
$\underit{C}$ is negative. The following proposition shows that these two
properties are related:
\begin{proposition}{\label{PropFocalPointTS}}
    Assume $S$ to be a closed two-dimensional surface (not necessarily
    trapped) in a Lorentzian manifold $\left(\mathcal{M}, g\right)$ which
    satisfies the Einstein equation $\mathrm{Ric}\left(g\right) = 0$. If
    $\mathrm{tr}\chits < 0$ at some point $x\in S$, then $\exists$ a focal point
    on the null generator $G_x$ of $C$ emanating from point $x$. A similar
    result holds for $\underit{C}$.
\end{proposition}
\begin{proof}[Proof to Proposition \ref{PropFocalPointTS}]
    Using Raychaudhari equation\footnote[5]{\hspace{0.3em}For our case, the null lapse function, $\Omega = 1$, hence $\omega = 0$.},
    \begin{equation*}
        \cancel{\nabla}_4\left(\mathrm{tr}\chits\right) = -\left|{\chits}\right|^2 + \omega\mathrm{tr}\chits
    \end{equation*}
    and \textcolor{red}{the other equation:}
    \begin{equation*}
        \left|\chits\right|^2 = \frac{1}{2}\left(\mathrm{tr}\chits\right)^2 + |\hat{\chits}|^2
    \end{equation*}
    we get, the following Riccati-type equation\footnote[4]{\hspace{0.3em}Riccati-type equations have a finite-time blow-up.}:
    \begin{equation}{\label{eq:Riccatieq}}
        \nabla_L\left(\mathrm{tr}\chits\right) = -\frac{1}{2}\left(\mathrm{tr}\chits\right)^2 - |\hat{\chits}|^2 \leq 0
    \end{equation}
    If $T = \mathrm{tr}\chits_x = \mathrm{tr}\chits\hspace{-0.2em}\left(0\right) < 0$, then
    $\mathrm{tr}\chits\hspace{-0.2em}\left(\tau\right) < 0\;, \hspace{0.8em} \forall \tau \geq 0$. \\
    Ignoring the term $|\hat{\chits}|$, yields:
    \begin{equation*}
        \nabla_L\left(-\frac{1}{\mathrm{tr}\chits}\right) \leq -\frac{1}{2}
    \end{equation*}
    Thus,
    \begin{align*}
        \scalebox{0.7}{$\implies$} & \nabla_L\left(-\frac{1}{\mathrm{tr}\chits}\right) \leq -\frac{1}{2}\\[0.8em]
        \scalebox{0.7}{$\implies$} & -\frac{1}{\mathrm{tr}\chits} \leq -\frac{1}{T} - \frac{\tau}{2}\\[0.8em]
        \scalebox{0.7}{$\implies$} & \tau_* = \frac{2}{-T} = \frac{2}{-\mathrm{tr}\chits_x}
    \end{align*}
    We obtain $\mathrm{tr}\chits\left(\tau_*\right) = -\infty$ and hence $G_x\left(\tau_*\right)$ is the first focal point on $G_x$.
\end{proof}
{\bfseries{Remarks:}} The equality in equation \eqref{eq:Riccatieq} holds for
$\mathrm{Ric}\left(L, L\right) = \mathrm{tr}\;\alpha = 0$. From second
variational formula \textcolor{red}{insert equation no.}, we see that
Riccati-type equation similar to equation\ref{eq:Riccatieq} can be obtained for
a more relaxed condition, $\mathrm{tr}\alpha = \mathrm{Ric}\left(L, L\right)\geq
0$. This is also called as the \textit{positive null energy condition}, which is
weaker than the Einstein equations. Hence, the proposition
\ref{PropFocalPointTS} also holds for the positive null energy condition.



%================== PENROSE INCOMPLETENESS THEOREM ==========================
\section{Penrose Incompleteness Theorem}
\subsection{Motivation}
\begin{theorem}[Penrose Incompleteness Theorem]\label{incompletethm}
Let $\left(\mathcal{M}, g\right)$ be a globally hyperbolic
time-orientable (Hausdorff) spacetime with a non-compact Cauchy Hypersurface
$\mathcal{H}$ such that $\mathcal{M}$ contains a trapped surface $S$. If, in
addition, $\left(\mathcal{M}, g\right)$ satisfies $\text{\textit{Ric}}\left(L,
L\right) \geq 0 \;\forall$ null vector fields $L$, then $\mathcal{M}$ is future
geodesically incomplete. In fact, $\exists$ a null generator of $C \cup
\textit{\underbar{C}}$, the future null geodesic congruences normal to $S$, that cannot
be extended $\forall \tau \geq 0$ in $\mathcal{M}$.
\end{theorem}

\begin{corollary}{\label{MinkGeodcomplete}} 
    There are no trapped surfaces in Minkowski spacetime as it is geodesic complete.
\end{corollary}

\begin{definition}[Geodesic Complete]
    A geodesic complete manifold implies that all its causal geodesics can be
    extended to arbitrary values of their affine parameters. Formally, a
    manifold $\left(\mathcal{M}, g\right)$ is geodesically complete if every
    timelike (or null) geodesic $\gamma : \mathbb{R}\supset \left(-\varepsilon,
    \varepsilon\right) \to \mathcal{M}$, such that $\gamma\left(0\right) = p,
    \;\forall p$ can be extended to $\widetilde{\gamma} : \mathbb{R} \to
    \mathcal{M}$. That is, the timelike or null geodesics can be extended from a
    real interval to the entire real line. 
\end{definition}

\begin{proof}[Proof to Corollary \ref{MinkGeodcomplete}] 
    In the cartesian coordinate frame for the spatial coordinates, the Minkowski
    metric, $\eta$, takes the form, $\eta = -\textit{dt}^2 + \textit{dx}^2 +
    \textit{dy}^2 + \textit{dz}^2$. The Christoffel symbols,
    $\Gamma^{\sigma}_{\alpha\beta}$, $\alpha, \beta, \sigma = \left\{0, 1, 2,
    3\right\}$, for this coordinate choice are all identically zero. The geodesic
    equation for an affinely parameterized curve, $\gamma\left(\tau\right)$,
    with $\tau$ as the affine parameter, takes the form:
    \begin{equation*}
        \frac{d^2 x^\sigma}{d\tau^2} + \Gamma^{\sigma}_{\alpha\beta}\frac{dx^\alpha}{d\tau}\dfrac{dx^\beta}{d\tau} = 0
    \end{equation*}
    , which reduces to {\large{$\frac{d^2 x^\sigma}{d\tau^2} = 0$}}. The general
    solution of such a differential equation is given by
    $x^{\mu}\left(\tau\right) = A^\mu \tau + B^\mu$, where the constants $A^\mu$
    and $B^\mu$ can be determined based on the initial data. We see that
    $x^\mu\left(\tau\right)$ is linear in the affine parameter and can be
    extended to any arbitrary value of $\tau$. Since this is true for any
    arbitrary curve in Minkowski spacetime, this completes our proof.
\end{proof}

The proof of the Penrose Incompleteness theorem relies on two important
theorems/results from differential topology. These are presented below and proof
to only the second one is provided:
\begin{enumerate}[leftmargin=*, label={\bfseries{Ingredient theorem (\roman*)}}]
	\item \label{topologicaltheorem1} A bijective continuous map from compact
	spaces to a Hausdorff space is a homeomorphism.
	\item \label{topologicaltheorem2} Let $\mathcal{N} \subset \mathcal{M}$ be
	an (injective) immersed topological submanifold with
	dim$\left(\mathcal{N}\right) = $ dim$\left(\mathcal{M}\right) = n$ such that
	$\mathcal{N}$ is compact and $\mathcal{M}$ is a Hausdorff connected
	non-compact topological manifold. Then $\partial_{\text{mani}}\mathcal{N} \neq
	\varnothing$, where $\partial_{\text{mani}}\mathcal{N}$ denotes the boundary
	of $\mathcal{N}$ in the sense of (topological) manifolds.
\end{enumerate}

\begin{proof}[Proof to \ref{topologicaltheorem2} (By contradiction)]
    Assume $\partial_{\text{mani}}\mathcal{N} = \varnothing$ \newline $\implies
    \forall x \in \mathcal{N}, \;\exists \delta > 0$, such that
    $\text{B}_\delta\left(x\right)$ homeomorphic to $\mathbb{R}^n$.\newline
    Since, dim$\left(\mathcal{M}\right) = \text{dim}\left(\mathcal{N}\right)$,
    $\text{B}_\delta\left(x\right)$ is also open in $\mathcal{M}$. By
    compactness, we can say that $\mathcal{N} \subset \mathcal{M}$ is open. However,
    $\mathcal{N}$ is compact and is a subset of a Hausdorff space, hence,
    $\mathcal{N}$ is closed. By connectedness, $\mathcal{M} = \mathcal{N}$,
    which is a contradiction.
\end{proof}

We move on to the proof of the Penrose Incompleteness theorem.
\begin{proof}[Proof to the Penrose Incompleteness Theorem \ref{incompletethm}]
From the results of \textcolor{red}{Section - look into the previous section},
we know that if $\mathrm{tr}\raisebox{0.5ex}{$\chi$}_x = -k_x < 0, \;x \in S =
\{\tau = 0\}$, then the first focal point of the generator $G_x \subset C$
appears at time {\large{$\tau = \frac{2}{k_x}$}}. In view of compactness of
$S$\footnote[2]{A continuous function on a compact set is bounded and attains its
maximum}, we have:
\begin{equation}
    \sup_{S} \mathrm{tr}\raisebox{0.5ex}{$\chi$} = k_{C} < 0\;, \qquad
    \sup_{S} \mathrm{tr}\raisebox{0.5ex}{$\chi$} = k_\textit{\underbar{C}} < 0\;, 
    \qquad \sup \{k_C, k_\textit{\underbar{C}}\} = k < 0
\end{equation}
We assume $\left(\mathcal{M}, g\right)$ to be future null geodesically complete.
We define $\mathscr{V}$ to be the union of all the null generators of $C \cup
\textit{\underbar{C}}$ for which $0 \leq \tau \leq 2/k$, i.e.,
\begin{equation}\
    \mathscr{V} = \bigcup_{\substack{\tau \;\in \;\left[0, \;\frac{2}{k}\right]\;, \\[0.2em] x \;\in \;S}}
    \left(G_x\left(\tau\right), \;\; \textit{\underbar{G}}_x\left(\tau\right)\right)
\end{equation}
\textcolor{red}{(Check this bit)} Based on this construction, $\mathscr{V}
\subset \mathcal{M}$. The null generators, $G$ and \textit{\underbar{G}}, can be
viewed as the following continuous maps:
\begin{equation*}
    \begin{split}
    G: S \times \left[0, \frac{2}{k}\right] \to C\;, \qquad \left(x, \tau\right) \erelbar{21} G_x\left(\tau\right) \\
    \qquad \textit{\underbar{G}}: S \times \left[0, \frac{2}{k}\right] \to \textit{\underbar{C}}\;, \qquad \left(x, \tau\right) \erelbar{21} \textit{\underbar{G}}_x\left(\tau\right) 
    \end{split}
\end{equation*}
Continuity implies that the null generators are mappings between compact sets.
This implies that $\mathscr{V}$ is itself compact.\\
The trace condition implies every null generator in $\mathscr{V}$ contains at
least one focal point. Therefore, \textcolor{red}{from the previous section on
causality}, we can say that:
\begin{equation}{\label{eq:CausalBoundaryVM}}
    \partial\hspace{-0.3em}\mathcal{J}^+\left(S\right) \subseteq \mathscr{V} \subset \mathcal{M}.
\end{equation}
The topological boundary $\partial\hspace{-0.3em}\mathcal{J}^+\left(S\right)$ is
closed by definition\footnote[3]{Topological boundary of subset $A$ of a
topological space $X$, $\partial\hspace{-0.2em} A$, is given by
$\partial\hspace{-0.2em} A = \overline{A} \cap \overline{X \backslash A}$.}.
Since $\mathscr{V}$ is compact, by \ref{eq:CausalBoundaryVM}, we conclude
$\partial\hspace{-0.3em}\mathcal{J}^+\left(S\right)$ is
compact\footnote[7]{Closed subset of a compact set is itself compact.}.\\
We now show that global topological argument leads to a contradiction, and hence
$\left(\mathcal{M}, g\right)$ cannot be future null geodesically complete.

Since $\mathcal{M}$ is time-orientable, $\exists$ global timelike vector field
$T$ whose integral curves are timelike foliate of $\mathcal{M}$  and intersect
the Cauchy Hypersurface exactly once. Futhermore, the integral curves intersect
$\partial\hspace{-0.3em}\mathcal{J}^+\left(S\right)$ exactly once, since
$\mathcal{J}^+\left(S\right)$ is future set and the topological boundary of a
future set is a closed achronal three-dimensional Lipschitz submanifold without
boundary\footnote[9]{\textbf{Proposition:} Let $\mathcal{F}$ be a future set in
a Lorentzian manifold $\mathcal{M}$. Then the topological boundary
$\partial\hspace{-0.2em}\mathcal{F}$ is closed achronal three-dimensional
locally Lipschitz submanifold of $\mathcal{M}$ such that
$\partial_{\text{mani}}\partial\hspace{-0.2em}\mathcal{F} \neq \varnothing$}
where $\partial_{\text{mani}}$ denotes the boundary in the sense of
(topological) manifolds. Projection of
$\partial\hspace{-0.3em}\mathcal{J}^+\left(S\right)$ on $\mathcal{H}$ via the
integral curve $T$ is a continuous injective mapping from
$\partial\hspace{-0.3em}\mathcal{J}^+\left(S\right)$ onto a subset of
$\mathscr{T}\subset\mathcal{H}$. \\
From \ref{topologicaltheorem1}, we can say that
$\partial\hspace{-0.3em}\mathcal{J}^+\left(S\right)$ is homeomorphic to
$\mathscr{T}\subset\mathcal{H}$. This implies that $\mathscr{T}$ is a Lipschitz
three-dimensional compact submanifold without a boundary, i.e.,
$\partial_\text{mani}\mathscr{T}\neq\varnothing$, in the compact
three-dimensional manifold $\mathcal{H}$. However, from
\ref{topologicaltheorem2}, $\mathscr{T}$ must have a non-empty boundary. This is
a contradiction. Thus, our assumption that $\left(\mathcal{M}, g\right)$ is
future null geodesically complete is false. 
\end{proof}

\end{document}