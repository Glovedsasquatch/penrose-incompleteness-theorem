\documentclass[11pt, a4paper]{report}
    \usepackage[utf8]{inputenc}
    \usepackage[T1]{fontenc}
    \usepackage[urw-garamond]{mathdesign} %for using math with garamond font
    %\usepackage{lmodern}
    \usepackage{titlesec}
    \usepackage{amsmath} %for cross-referencing math equations
    \usepackage[mathscr]{eucal}
    %\usepackage{amsfonts}
    \usepackage{amsthm}
    \usepackage{bm} %for using bold font letters in inline mathematical expressions
    \usepackage{graphicx}
    \usepackage{graphics}
    \usepackage{subcaption}
    \usepackage{array}
    \usepackage{epstopdf}
    \usepackage{titlesec} %for numbering of subsubsection
    \usepackage{parskip}
    \usepackage[symbol]{footmisc} % for using footnote with symbols instead of numbers
    \usepackage{enumitem} %for enumerations of lists with numbers or alphabets and so on.......
    \usepackage[margin=2cm, includefoot]{geometry}
    \usepackage[labelfont={small, bf}, textfont={small, it}]{caption} %used for making the captions in images small and italicized
    
    % Packages for title page and section formatting
    \usepackage{titling}
    \usepackage{sectsty}
    \usepackage{titlesec}

    % Page Layout
    \textwidth 165mm    % also can be set to 155mm///// 158mm
    \textheight 238mm   % standard
    \topmargin -10mm

    \oddsidemargin 0.1cm
    \evensidemargin -0.07cm
    \graphicspath{ {/images/} }

    % Stopping Figures from Using a Whole Page
    \renewcommand{\topfraction}{0.95}
    \renewcommand{\textfraction}{0.05}
    \renewcommand{\floatpagefraction}{0.85}
    \renewcommand{\baselinestretch}{1.5}
    \renewcommand*\descriptionlabel[1]{\hspace\leftmargin$#1$}
    \renewcommand{\thefootnote}{\fnsymbol{footnote}} %for using footnotes with symbols rather than numbers
    \renewcommand{\bibname}{References} %to rename bibliography section to "References"

    
    % Set section font size and type, and adjust spacing
    \titleformat{\section}
        {\Large\bfseries\sffamily} % Customize font size and type
        {\thesection}{1em}{}
    \titlespacing*{\section}
        {0pt}{3.0ex plus 1ex minus .2ex}{2.0ex plus .2ex}

    \titleformat{name=\section, numberless}
        {\Large\bfseries\sffamily} % Customize font size and type
        {}{0pt}{}
        
    % Optionally, customize subsection and subsubsection as well
    \titleformat{\subsection}
        {\large\bfseries\itshape}
        {\thesubsection}{1em}{}
    \titlespacing*{\subsection}
        {0pt}{2.0ex plus 1ex minus .2ex}{1.0ex plus .2ex}

    \titleformat{\subsubsection}
        {\normalsize\bfseries\scshape}
        {\thesubsubsection}{1em}{}
    \titlespacing*{\subsubsection}
        {0pt}{1.5ex plus 1ex minus .2ex}{1.0ex plus .2ex}
       
    %defines a new format for the theorems (such as top space, bottom space, text title type, etc.)
    \newtheoremstyle{bfnote}%
        {16.0pt plus 2.0pt minus 4.0pt}{\topsep} %\topsep is defined as {8.0pt plus 2.0pt minus 4.0pt}
        {\itshape}{}
        {\bfseries}{.}
        { }{\thmname{#1}\thmnumber{ #2}\thmnote{ (#3)}}

    \theoremstyle{bfnote}
    
    \newtheorem{example}{Example}[section] %for enumerating examples within a chapter
    \newtheorem{theorem}{Theorem}[section]
    \newtheorem{corollary}{Corollary}[theorem]
    \newtheorem{definition}{Definition}
    
    \renewcommand{\thesection}{\arabic{section}} %to remove 0.x in numbered sections
    %\renewcommand{\theorem}{\arabic{section}.\arabic{theorem}}


\begin{document}
%============================================BEGIN TITLE PAGE===============================================
\begin{titlepage}
    \begin{center}
    %syntax for entering line \line(slope){length} double backslash for change of line. For changing
    %line one can even leave a space. 
    \line(1,0){300}\\
    %for increasing the space between the line and the title [spacing]	
    [0.2in]
    \normalsize{\emph {Seminar Report}}\\	
    \Huge{\bfseries Penrose Incompleteness Theorem}\\
    [-0.12in]
    \line(1,0){300}\\
    
    \vspace{3in}
    
    \LARGE{\bfseries {Prabha Shankar}}\\
    \small{Masters Student, Mathematical Physics}\\
    \small{(Department of Mathematics, Leipzig University)}\\
    
    \vspace{3cm}
    \normalsize{\emph {Under the supervision of}}\\
    \large{\bfseries {Prof. Dr. Stephan Czimek}}\\
    \small{(Department of Mathematics, Leipzig University)}\\
    
    \end{center}
\end{titlepage}

%=================TABLE OF CONTENTS AND LIST OF FIGURES=======================
\section*{Introduction}
Entities composing the nature constantly alter due to the processes of nature
and their interactions with the other entities. The concept of conservation, an
omnipresent phenomena, is therefore the key to understand these processes as it
helps in the determination of parameters such as quantity of interacting
material, its velocity, temperature, energy and so forth. This report deals with
the nature of connservation laws and by means of mathematical modelling helps in
qualitative understanding of these equations as propagating waves of
disturbances through a medium. Specifically, this report is an endeavor to
understand the challenges that arise when numerically solving Hyperbolic PDEs by
throwing light on the discontunuities that arise in its solution space and
brings forth key ideas and techniques to better combat the challenges in their
computational implementations. 

\section{Double Null Hypersurfaces and Foliations}
This is the first chapter

\section{Null Structure Equations}
This is the second chapter

\section{Null Hypersurface Regularity}
This is the third chapter

\section{Causality for Spacetimes with Trapped Surfaces}
Obviously! This is the fourth chapter

\section{Penrose Incompleteness Theorem}
\subsection{Motivation}
\begin{theorem}[Penrose Incompleteness Theorem]
Let $\left(\mathcal{M}, g\right)$ be a globally hyperbolic
time-orientable (Hausdorff) spacetime with a non-compact Cauchy Hypersurface
$\mathcal{H}$ such that $\mathcal{M}$ contains a trapped surface $S$. If, in
addition, $\left(\mathcal{M}, g\right)$ satisfies $\text{\textit{Ric}}\left(L,
L\right) \geq 0 \;\forall$ null vector fields $L$, then $\mathcal{M}$ is future
geodesically incomplete. In fact, $\exists$ a null generator of $C \cup
\underbar{$C$}$, the future null geodesic congruences normal to $S$, that cannot
be extended $\forall \tau \geq 0$ in $\mathcal{M}$.
\end{theorem}

\begin{corollary}{\label{MinkGeodcomplete}} 
    There are no trapped surfaces in Minkowski spacetime as it is geodesic complete.
\end{corollary}

\begin{definition}[Geodesic Complete]
    A geodesic complete manifold implies that all its causal geodesics can be
    extended to arbitrary values of their affine parameters. Formally, a
    manifold $\left(\mathcal{M}, g\right)$ is geodesically complete if every
    timelike (or null) geodesic $\gamma : \mathbb{R}\supset \left(-\epsilon,
    \epsilon\right) \to \mathcal{M}$, such that $\gamma\left(0\right) = p,
    \;\forall p$ can be extended to $\widetilde{\gamma} : \mathbb{R} \to
    \mathcal{M}$. That is, the timelike or null geodesics can be extended from a
    real interval to the entire real line.
\end{definition}

\begin{proof}[Proof to Corollary \ref{MinkGeodcomplete}] 
    In the cartesian coordinate frame for the spatial coordinates, the Minkowski
    metric, $\eta$, takes the form, $\eta = -\textit{dt}^2 + \textit{dx}^2 +
    \textit{dy}^2 + \textit{dz}^2$. The Christoffel symbols,
    $\Gamma^{\sigma}_{\alpha\beta}$, $\alpha, \beta, \sigma = \left\{0, 1, 2,
    3\right\}$, for this coordinate choice are all identically zero. The geodesic
    equation for an affinely parameterized curve, $\gamma\left(\tau\right)$,
    with $\tau$ as the affine parameter, takes the form:
    \begin{equation*}
        \frac{d^2 x^\sigma}{d\tau^2} + \Gamma^{\sigma}_{\alpha\beta}\frac{dx^\alpha}{d\tau}\dfrac{dx^\beta}{d\tau} = 0
    \end{equation*}
    , which reduces to {\large{$\frac{d^2 x^\sigma}{d\tau^2} = 0$}}. The general solution of
    such a differential equation is given by $x^{\mu}\left(\tau\right) = A^\mu
    \tau + B^\mu$, where the constants $A^\mu$ and $B^\mu$ can be determined
    based on the initial data. We see that $x^\mu\left(\tau\right)$ is linear in
    the affine parameter and can be extended to any arbitrary value of $\tau$.
    Since this is true for any arbitrary curve in Minkowski spacetime, this
    completes our proof.
\end{proof}

\end{document}
